\documentclass[12pt,a4paper]{article}
\usepackage[margin=1.2in]{geometry}
\usepackage{ctex}
\usepackage{graphicx}
\usepackage{enumitem}
\usepackage{multirow}
\usepackage{booktabs}
\usepackage{hyperref}

% 配置图表编号格式
\graphicspath{{figures/}} % 图表文件存放路径
\renewcommand{\figurename}{图}
\renewcommand{\tablename}{表}

% 文档标题设置
\title{计算机组成原理 Attack Lab 实验报告}
\author{学号:2023010304\quad 姓名:李孙木铎\quad 班级:软件31}
\date{\today}

\begin{document}
\maketitle
\tableofcontents % 生成目录
\newpage

\section{实验概述}
\subsection{实验背景与目的}
\paragraph{背景说明}
Attack Lab 围绕缓冲区溢出漏洞展开,通过对两个存在安全缺陷的可执行程序(\texttt{ctarget} 和 \texttt{rtarget})设计攻击输入,理解攻击者利用漏洞的原理,同时掌握程序运行时栈结构、指令编码及系统安全防护机制。

\paragraph{实验目标}
\begin{enumerate}[label=(\arabic*)]
    \item 深入理解 x86-64 架构下栈结构与函数调用参数传递机制;
    \item 掌握代码注入(Code Injection, CI)攻击的原理与实现方法;
    \item 掌握返回导向编程(Return-Oriented Programming, ROP)攻击的原理与实现方法;
    \item 熟悉调试工具(GDB、OBJDUMP)与攻击字符串生成工具(HEX2RAW)的使用;
    \item 理解编译器与操作系统的安全防护机制(如栈随机化、栈不可执行)。
\end{enumerate}

\subsection{实验环境与工具}
\subsubsection{环境配置}
\begin{itemize}
    \item 操作系统:\underline{\qquad\qquad\qquad}(建议 x86-64 Ubuntu 22.04 或助教提供虚拟机)
    \item 架构:x86-64
    \item 关键配置:\underline{\qquad\qquad\qquad}(如是否关闭地址空间随机化 ASLR)
\end{itemize}

\subsubsection{核心工具}
\begin{table}[htbp]
    \centering
    \begin{tabular}{|c|c|c|}
        \hline
        工具名称 & 用途 & 关键命令示例 \\
        \hline
        \texttt{OBJDUMP} & 反汇编可执行文件,获取指令与地址 & \texttt{objdump -d ctarget > ctarget.dis} \\
        \hline
        \texttt{GDB} & 程序调试(断点、寄存器/内存查看) & \texttt{gdb ctarget}、\texttt{break *0x400xxx} \\
        \hline
        \texttt{HEX2RAW} & 将十六进制攻击字符串转为原始字节流 & \texttt{./hex2raw < exploit.txt | ./ctarget} \\
        \hline
        \texttt{tar} & 解压实验文件包 & \texttt{tar -xvf targetk.tar} \\
        \hline
    \end{tabular}
    \caption{实验核心工具说明}
    \label{tab:tools}
\end{table}

\subsection{实验文件说明}
从 \url{http://nuc.kirotta.top:15513/} 获取的 \texttt{targetk.tar} 解压后包含以下文件:
\begin{itemize}
    \item \texttt{README.txt}:目录文件说明;
    \item \texttt{ctarget}:存在代码注入漏洞的可执行程序;
    \item \texttt{rtarget}:存在 ROP 漏洞的可执行程序;
    \item \texttt{cookie.txt}:个人专属 8 位十六进制标识(攻击参数核心);
    \item \texttt{farm.c}:ROP 攻击所需的“gadget 农场”源代码;
    \item \texttt{hex2raw}:攻击字符串转换工具。
\end{itemize}

\subsection{实验任务总览}
实验分为 5 个阶段,其中前 4 个阶段计分,第 5 阶段为可选任务:
\begin{table}[htbp]
    \centering
    \begin{tabular}{|c|c|c|c|c|c|}
        \hline
        阶段 & 目标程序 & 攻击类型 & 目标函数 & 核心要求 & 分值 \\
        \hline
        1 & \texttt{ctarget} & 代码注入(CI) & \texttt{touch1} & 重定向返回地址至 \texttt{touch1} & 10 \\
        \hline
        2 & \texttt{ctarget} & 代码注入(CI) & \texttt{touch2} & 传递 \texttt{cookie} 作为参数 & 25 \\
        \hline
        3 & \texttt{ctarget} & 代码注入(CI) & \texttt{touch3} & 传递 \texttt{cookie} 字符串地址作为参数 & 25 \\
        \hline
        4 & \texttt{rtarget} & 返回导向编程(ROP) & \texttt{touch2} & 用 gadget 传递 \texttt{cookie} 参数 & 35 \\
        \hline
        5(可选) & \texttt{rtarget} & 返回导向编程(ROP) & \texttt{touch3} & 用 gadget 传递 \texttt{cookie} 字符串地址 & 0 \\
        \hline
    \end{tabular}
    \caption{实验任务与分值分布}
    \label{tab:tasks}
\end{table}


\newpage
\section{代码注入攻击(针对 ctarget)}
\subsection{阶段 1:跳转至 touch1 函数}
\subsubsection{实验原理}
\paragraph{函数调用栈结构}
\texttt{getbuf} 由 \texttt{test} 函数调用,正常执行流程为:\texttt{test -> getbuf -> ret 回到 test}。攻击需修改 \texttt{getbuf} 返回时的栈中“返回地址”,使其指向 \texttt{touch1} 函数入口地址。

\paragraph{关键问题}
\begin{enumerate}[label=(\arabic*)]
    \item 确定 \texttt{getbuf} 中缓冲区 \texttt{buf} 的大小(\texttt{BUFFER\_SIZE});
    \item 确定栈中“返回地址”相对于 \texttt{buf} 的偏移量;
    \item 获取 \texttt{touch1} 函数的入口地址。
\end{enumerate}

\subsubsection{实验步骤}
1. \underline{\qquad\qquad\qquad}(如用 OBJDUMP 反汇编 \texttt{ctarget},分析 \texttt{getbuf} 栈帧);\\
2. \underline{\qquad\qquad\qquad}(如用 GDB 调试确定缓冲区偏移量);\\
3. \underline{\qquad\qquad\qquad}(获取 \texttt{touch1} 入口地址);\\
4. \underline{\qquad\qquad\qquad}(构造攻击字符串,用 HEX2RAW 转换并测试)。

\subsubsection{攻击字符串设计}
\paragraph{字符串结构}
\begin{verbatim}
# 格式:[填充字节(BUFFER_SIZE 字节)][touch1 入口地址(8 字节,小端序)]
示例:
00 00 00 00 00 00 00 00  # 填充字节(假设 BUFFER_SIZE=8,实际需按反汇编结果调整)
00 00 00 00 04 08 00 00  # touch1 地址(示例,需替换为实际地址,小端序)
\end{verbatim}

\paragraph{最终攻击字符串(hex 格式)}
\underline{\qquad\qquad\qquad}

\subsubsection{实验结果与验证}
\paragraph{执行命令}
\texttt{./hex2raw < ctarget01.txt | ./ctarget}

\paragraph{预期输出}
\begin{verbatim}
Cookie: 0xXXXXXXXXX  # 个人 cookie
Type string:Touch1!: You called touch1()
Valid solution for level 1 with target ctarget
PASSED: Sent exploit string to server to be validated.
NICE JOB!
\end{verbatim}

\paragraph{实际结果与问题分析}
\underline{\qquad\qquad\qquad}


\subsection{阶段 2:跳转至 touch2 函数(传递 cookie 参数)}
\subsubsection{实验原理}
\paragraph{参数传递规则}
x86-64 架构下,函数第一个参数通过 \%rdi 寄存器传递。攻击需:
1. 在栈中注入自定义汇编代码,将 \texttt{cookie} 加载到 \%rdi;
2. 修改 \texttt{getbuf} 返回地址,使其先跳至注入代码,再由注入代码跳至 \texttt{touch2}。

\paragraph{核心汇编指令设计}
需实现:\texttt{mov \$cookie, \%rdi; ret;}(将 cookie 存入 \%rdi,再跳至 \texttt{touch2})。

\subsubsection{实验步骤}
1. \underline{\qquad\qquad\qquad}(反汇编 \texttt{ctarget},确定 \texttt{buf} 地址与返回地址偏移);\\
2. \underline{\qquad\qquad\qquad}(编写汇编代码,生成对应机器码);\\
3. \underline{\qquad\qquad\qquad}(确定注入代码在栈中的地址,即 \texttt{buf} 起始地址);\\
4. \underline{\qquad\qquad\qquad}(构造攻击字符串:填充字节 + 注入代码地址 + 注入代码 + cookie);\\
5. \underline{\qquad\qquad\qquad}(用 HEX2RAW 转换并测试)。

\subsubsection{攻击字符串设计}
\paragraph{字符串结构}
\begin{verbatim}
# 格式:[填充字节(偏移量 - 注入代码长度 字节)][注入代码地址(8 字节)][注入代码(机器码)][cookie(8 字节,可选,视指令设计)]
示例:
00 00 00 00 00 00 00 00  # 填充字节(使返回地址指向注入代码)
10 20 00 00 00 00 00 00  # 注入代码地址(即 buf 起始地址,小端序)
48 c7 c7 XX XX XX XX 00  # 注入代码:mov $0xXXXXXXXX, %rdi(XX 为 cookie 字节,小端序)
c3 00 00 00 00 00 00 00  # 注入代码:ret(跳至 touch2,需在指令中指定 touch2 地址)
\end{verbatim}

\paragraph{汇编代码与机器码转换}
\begin{table}[htbp]
    \centering
    \begin{tabular}{|c|c|c|}
        \hline
        汇编指令 & 机器码(hex) & 说明 \\
        \hline
        \texttt{mov \$0xXXXXXXXXX, \%rdi} & \underline{\qquad\qquad\qquad} & 将 cookie 存入 \%rdi \\
        \hline
        \texttt{jmp touch2} 或 \texttt{ret} & \underline{\qquad\qquad\qquad} & 跳至 touch2(ret 需配合栈中地址) \\
        \hline
    \end{tabular}
    \caption{注入代码的汇编与机器码对应关系}
    \label{tab:asm2machine}
\end{table}

\paragraph{最终攻击字符串(hex 格式)}
\underline{\qquad\qquad\qquad}

\subsubsection{实验结果与验证}
\paragraph{执行命令}
\texttt{./hex2raw < ctarget02.txt | ./ctarget}

\paragraph{预期输出}
\begin{verbatim}
Cookie: 0xXXXXXXXXX
Type string:Touch2!: You called touch2(0xXXXXXXXXX)
Valid solution for level 2 with target ctarget
PASSED: Sent exploit string to server to be validated.
NICE JOB!
\end{verbatim}

\paragraph{实际结果与问题分析}
\underline{\qquad\qquad\qquad}


\subsection{阶段 3:跳转至 touch3 函数(传递 cookie 字符串参数)}
\subsubsection{实验原理}
\paragraph{参数要求}
\texttt{touch3} 需接收指向 \texttt{cookie} 字符串(8 位 hex + 末尾 null 字节)的地址作为参数。关键挑战:
1. 字符串需存入栈中,且避免被 \texttt{hexmatch} 等函数调用时的栈操作覆盖;
2. 注入代码需将字符串地址加载到 \%rdi,再跳至 \texttt{touch3}。

\paragraph{字符串存储位置选择}
建议将字符串存于“返回地址之后”的栈空间(避免被函数调用覆盖),即攻击字符串结构包含:[填充] + [注入代码地址] + [注入代码] + [cookie 字符串]。

\subsubsection{实验步骤}
1. \underline{\qquad\qquad\qquad}(确定字符串在栈中的存储地址);\\
2. \underline{\qquad\qquad\qquad}(编写注入代码:将字符串地址存入 \%rdi,跳至 \texttt{touch3});\\
3. \underline{\qquad\qquad\qquad}(构造攻击字符串:填充 + 注入地址 + 注入代码 + cookie 字符串 + null 字节);\\
4. \underline{\qquad\qquad\qquad}(测试并调试字符串地址正确性)。

\subsubsection{攻击字符串设计}
\paragraph{字符串结构}
\begin{verbatim}
# 格式:[填充字节][注入代码地址][注入代码][cookie 字符串(8 字节 ASCII)][00(null 终止符)]
示例:
00 00 00 00 00 00 00 00  # 填充字节
10 20 00 00 00 00 00 00  # 注入代码地址(buf 起始地址)
48 c7 c7 30 20 00 00 00  # 注入代码:mov $0x2030, %rdi(字符串地址,示例)
c3 31 32 33 34 35 36 37  # ret 指令 + cookie 字符串(假设 cookie 为 12345678)
38 00 00 00 00 00 00 00  # 字符串剩余部分 + null 终止符
\end{verbatim}

\paragraph{Cookie 字符串转换}
例如:cookie 为 0x1a7dd803 → 字符串为 "1a7dd803" → ASCII 码为 31 61 37 64 64 38 30 33。

\paragraph{最终攻击字符串(hex 格式)}
\underline{\qquad\qquad\qquad}

\subsubsection{实验结果与验证}
\paragraph{执行命令}
\texttt{./hex2raw < ctarget03.txt | ./ctarget}

\paragraph{预期输出}
\begin{verbatim}
Cookie: 0xXXXXXXXXX
Type string:Touch3!: You called touch3("XXXXXXXX")  # XXXXXXXX 为 cookie 字符串
Valid solution for level 3 with target ctarget
PASSED: Sent exploit string to server to be validated.
NICE JOB!
\end{verbatim}

\paragraph{实际结果与问题分析}
\underline{\qquad\qquad\qquad}


\newpage
\section{返回导向攻击(针对 rtarget)}
\subsection{阶段 2:ROP 攻击 touch2 函数(传递 cookie 参数)}
\subsubsection{实验原理}
\paragraph{RTARGET 安全机制}
\texttt{rtarget} 启用栈随机化与栈不可执行,代码注入失效。需通过“gadget”(现有代码中“指令序列 + ret”)串联执行,核心目标:将 \texttt{cookie} 存入 \%rdi 并跳至 \texttt{touch2}。

\paragraph{Gadget 需求}
需两类 gadget:
1. \texttt{popq \%rax; ret;}:从栈中弹出 \texttt{cookie} 至 \%rax;
2. \texttt{movq \%rax, \%rdi; ret;}:将 \%rax 中的 \texttt{cookie} 转移至 \%rdi。

\subsubsection{实验步骤}
1. \underline{\qquad\qquad\qquad}(反汇编 \texttt{rtarget},在 start\_farm ~ mid\_farm 间寻找目标 gadget);\\
2. \underline{\qquad\qquad\qquad}(记录 gadget 入口地址与 \texttt{touch2} 入口地址);\\
3. \underline{\qquad\qquad\qquad}(构造 ROP 链:填充字节 + gadget1 地址 + cookie + gadget2 地址 + touch2 地址);\\
4. \underline{\qquad\qquad\qquad}(用 HEX2RAW 转换并测试,多次执行验证栈随机化适应性)。

\subsubsection{攻击字符串(ROP 链)设计}
\paragraph{ROP 链结构}
\begin{verbatim}
# 格式:[填充字节(覆盖 getbuf 栈帧至返回地址)][gadget1 地址][cookie][gadget2 地址][touch2 地址]
# 说明:ret 指令会将栈顶值作为下一条指令地址,故 gadget 后需跟参数/下一个 gadget 地址
示例:
00 00 00 00 00 00 00 00  # 填充字节(按 getbuf 栈帧大小调整)
20 30 00 00 00 00 00 00  # gadget1:popq %rax; ret; 地址(小端序)
XX XX XX XX XX XX XX XX  # cookie(8 字节,小端序)
40 50 00 00 00 00 00 00  # gadget2:movq %rax, %rdi; ret; 地址(小端序)
60 70 00 00 00 00 00 00  # touch2 入口地址(小端序)
\end{verbatim}

\paragraph{Gadget 信息表}
\begin{table}[htbp]
    \centering
    \begin{tabular}{|c|c|c|c|}
        \hline
        Gadget 功能 & 指令序列 & 入口地址(hex) & 机器码(hex) \\
        \hline
        popq \%rax; ret; & \texttt{pop \%rax; ret;} & \underline{\qquad\qquad\qquad} & \underline{\qquad\qquad\qquad} \\
        \hline
        movq \%rax, \%rdi; ret; & \texttt{mov \%rax, \%rdi; ret;} & \underline{\qquad\qquad\qquad} & \underline{\qquad\qquad\qquad} \\
        \hline
    \end{tabular}
    \caption{ROP 攻击所用 Gadget 信息}
    \label{tab:gadgets}
\end{table}

\paragraph{最终攻击字符串(hex 格式)}
\underline{\qquad\qquad\qquad}

\subsubsection{实验结果与验证}
\paragraph{执行命令}
\texttt{./hex2raw < rtarget02.txt | ./rtarget}

\paragraph{预期输出}
\begin{verbatim}
Cookie: 0xXXXXXXXXX
Type string:Touch2!: You called touch2(0xXXXXXXXXX)
Valid solution for level 2 with target rtarget
PASSED: Sent exploit string to server to be validated.
NICE JOB!
\end{verbatim}

\paragraph{实际结果与问题分析}
\underline{\qquad\qquad\qquad}


\subsection{阶段 3:ROP 攻击 touch3 函数(可选,不计分)}
\subsubsection{实验原理}
\paragraph{核心挑战}
需传递 cookie 字符串地址至 \%rdi,需解决:
1. 栈随机化导致字符串绝对地址不可预测 → 利用 \%rsp 相对地址计算;
2. 需更多 gadget 实现地址计算(如 \texttt{addq \$offset, \%rsp; ret;})与寄存器中转。

\subsubsection{实验步骤}
\underline{\qquad\qquad\qquad}

\subsubsection{攻击字符串设计}
\underline{\qquad\qquad\qquad}

\subsubsection{实验结果与验证}
\underline{\qquad\qquad\qquad}


\newpage
\section{实验总结与反思}
\subsection{实验收获}
1. \underline{\qquad\qquad\qquad}(如对 x86-64 栈结构、指令编码的理解);\\
2. \underline{\qquad\qquad\qquad}(如调试工具使用技巧、攻击思路设计);\\
3. \underline{\qquad\qquad\qquad}(如系统安全防护机制的作用与局限性)。

\subsection{遇到的问题与解决方法}
\begin{table}[htbp]
    \centering
    \begin{tabular}{|c|c|c|}
        \hline
        问题描述 & 原因分析 & 解决方法 \\
        \hline
        阶段 1 触发段错误 & 地址小端序错误 / 填充字节长度不对 & 重新反汇编确认地址,调整填充长度 \\
        \hline
        阶段 2 \%rdi 未正确加载 cookie & 注入代码地址错误 / 机器码错误 & 用 GDB 查看寄存器值,验证机器码正确性 \\
        \hline
        ROP 攻击多次执行部分失败 & 栈随机化未适配 & 确认 gadget 地址是否在稳定区域,调整 ROP 链 \\
        \hline
        \underline{\qquad\qquad\qquad} & \underline{\qquad\qquad\qquad} & \underline{\qquad\qquad\qquad} \\
        \hline
    \end{tabular}
    \caption{实验问题与解决方案}
    \label{tab:problems}
\end{table}

\subsection{对课程的建议(可选)}
\underline{\qquad\qquad\qquad}

\end{document}