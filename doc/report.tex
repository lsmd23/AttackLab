\documentclass[11pt,a4paper]{article}
\usepackage{geometry}
\geometry{left=2.5cm,right=2.5cm,top=2.5cm,bottom=2.5cm}
\usepackage{fontspec}
\usepackage{xeCJK}
% \setCJKmainfont{SimSun} % 可按需改为系统中可用中文字体
% \usepackage{graphicx}
\usepackage{hyperref}
\usepackage{listings}
\usepackage{xcolor}
\usepackage{caption}
\usepackage{float}
\usepackage{amsmath}
\usepackage{booktabs}

\hypersetup{
    colorlinks=true,
    linkcolor=blue,
    citecolor=blue,
    urlcolor=cyan
}

\lstset{
  basicstyle=\ttfamily\small,
  keywordstyle=\color{blue}\bfseries,
  commentstyle=\color{gray},
  stringstyle=\color{red},
  breaklines=true,
  numbers=left,
  numberstyle=\tiny,
  frame=single,
  captionpos=b
}

\title{计算机组成原理课程实验 —— attacklab 报告}
\author{姓名:李孙木铎 \quad 学号:2023010304 \quad 班级:软件31}
\date{\today}

\begin{document}
\maketitle
\begin{abstract}
简要说明实验目的、实验环境和主要结果。示例:本次实验通过对 attacklab 的逆向与利用,掌握栈溢出/格式化字符串/返回导向等攻击原理,并编写/调试相应 payload,实现预期的利用效果。
\end{abstract}

\tableofcontents

\section{实验目的}
\begin{itemize}
  \item 理解并实践二进制漏洞利用的基本流程(分析、定位、构造 payload、执行验证)。
  \item 熟练使用调试器(gdb)、静态分析工具(objdump/strings/readelf)及动态跟踪工具(strace/ltrace)。
  \item 掌握绕过常见防护(ASLR/DEP/Canary/PIE)的思想与方法。
\end{itemize}

\section{实验环境}
\begin{itemize}
  \item 操作系统:Ubuntu XX.XX(内核 X.X)
  \item 编译器/工具链:gcc/clang, make
  \item 调试与分析工具:gdb, pwndbg/gef, objdump, readelf, nm, strace, ltrace
  \item 实验目录结构(示例):
  \begin{itemize}
    \item AttackLab/ -- 源码与可执行文件
    \item AttackLab/doc/ -- 报告与说明
    \item AttackLab/bin/ -- 编译产物
  \end{itemize}
\end{itemize}

\section{样例漏洞概述}
简要描述实验中目标二进制的功能与存在的漏洞类型(如:栈缓冲区溢出、格式化字符串、整数溢出等)。给出触发漏洞的输入示例。

\section{分析方法与步骤}
本节按步骤详细说明分析与利用过程。

\subsection{静态分析}
列出使用的静态分析命令和关键输出示例:
\begin{lstlisting}[language=bash,caption={示例命令}]
file target
readelf -h target
objdump -d target | less
nm target
strings target
\end{lstlisting}

\subsection{动态分析与调试}
说明 gdb 调试的关键断点、寄存器观察、内存布局分析方法;示例命令:
\begin{lstlisting}[language=bash,caption={gdb 常用操作}]
gdb -q target
(gdb) break main
(gdb) run < input
(gdb) info registers
(gdb) x/40x $rsp
\end{lstlisting}

\subsection{漏洞定位}
记录定位过程(函数、偏移、可控输入到返回地址的字节数)。给出确定偏移的实验与输出截图/文字描述。

\subsection{构造 Payload}
说明 payload 结构(填充、返回地址或ROP链、shellcode 等),并给出关键代码片段。例如使用 Python 构造 payload:
\begin{lstlisting}[language=python,caption={构造示例 payload}]
padding = b"A" * OFFSET
ret = p64(target_ret_addr)
payload = padding + ret
\end{lstlisting}

\subsection{绕过防护}
依次说明该二进制启用了哪些防护(NX/ASLR/Canary/PIE/RELRO),并描述采取的绕过策略。例如:
\begin{itemize}
  \item 当 NX 开启:采用 ROP 或 return-to-libc。
  \item 当 ASLR 开启:泄露地址后构造可重定位的 ROP 链或禁用 ASLR(仅用于测试)。
  \item 当 Canary 存在:泄露 canary 或寻找不受 canary 保护的函数。
\end{itemize}

\section{实验结果}
展示最终利用的步骤与运行结果(可包含程序输出、gdb 日志片段、成功获得 shell 的证明)。尽量附上关键截图或拷贝输出文本。

\section{安全分析与防护建议}
从开发者角度给出加固建议:
\begin{itemize}
  \item 使用栈保护(-fstack-protector-strong)、启用 PIE、RELRO、禁用不必要的函数(如 gets)。
  \item 采用地址空间布局随机化与现代分配器保护。
  \item 使用静态/动态检测手段(ASan/Valgrind/ fuzzing)提前发现类似缺陷。
\end{itemize}

\section{复现实验步骤(快速指南)}
列出从源码编译到重现 exploit 的简洁指令序列:
\begin{lstlisting}[language=bash,caption={示例复现命令}]
cd /home/thu/AttackLab
make clean && make
# 关闭ASLR(测试环境)
echo 0 | sudo tee /proc/sys/kernel/randomize_va_space
# 运行 exploit
python3 exploit.py
\end{lstlisting}

\section{结论与心得}
总结学到的关键点、遇到的难点与改进方向。

\appendix
\section{附录 A:关键源码/脚本}
将关键 C 源码、exploit 脚本或 patch 贴在此处(用 listings 环境展示)。

\section{附录 B:工具使用记录}
记录使用 gdb/pwndbg 的主要命令历史,便于他人复现。

\section{参考文献}
列出参考书目与在线资源(如 CSAPP、pwnable.tw、ROP 教程等)。

\end{document}